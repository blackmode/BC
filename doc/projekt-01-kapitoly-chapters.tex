%=========================================================================
% (c) Michal Bidlo, Bohuslav Křena, 2008

\chapter{Úvod}

V současné době se technologie ve všech odvětvích stále posouvá a není tomu jinak ani u prohlížení fotek a videí. V současné době i takto nahrané informace pomocí digitálních zařízení, je trend dále posouvat a zvyšovat tak zážitek ze zaznamenané události. Díky tomuto trendu si už nevystačíme s ``klasickými''  přehrávači, popř. prohlížeči videí, nebo fotografií. Díky sférickým 360° kamerám, je dnes možné zaznamenat video o srovnatelné velikosti jako u dnes běžného telefonu, ale s mnohem větším objemem informací, které se ale nedají srozumitelně přehrát v klasických přehrávačích. Ruku v ruce se sdílením takových dát, je velice výhodné takový prohlížeč nabídnout jako webovou verzi, aby i jiní uživatelé mohli svá takto natočená videa přehrávat, popř. sdílet.

Následující práce se věnuje implementaci webového prohlížeče panoramatických snímku a videí v různých modech, ve kterých je video interpretováno. Cílem je tedy navrhnout a zrealizovat řešení, aby uživatel po natočení videa dále nepotřeboval k přehrání např. sférického videa v režimu rybího oka specifický program, který by video musel nejprve překonvertovat. To stejné se týká i panoramatických snímku a videí v equirectangulárním modu.  

Další inovací v prohlížení videí je přidání některých dat, které v běžném prohlížení není k dispozici. Např. informace o světových stranách ve vztahu k obrázku, či videu popřípadě údaje o velikosti náhledu v daném kontextu prohlížení až po dodatečná metadata v~panoramatických obrázcích, které jsou vhodné např. k~tomu, aby autor obrázku mohl popsat jeho zajímavé částí apod.



\chapter{Webové technologie prohlížeče}

\section{HTML5}

Jazyk HTML (HyperText Markup Language) je značkovací jazyk určený pro popis webových stránek. Vychází z univerzálního značkovacího jazyka SGML (Standard Generalized Markup Language). V současné době aktuální verzi  jazyka HTML je jeho již pátá verze -  HTML5. Nová verze HTML přináší zásadní vylepšení, nové funkce a možnosti, které jsou nezbytné pro návrh a implementaci webového prohlížeče. 

Pro návrh samotného programu je nezbytná podpora multimedií – tedy audio a video, v neposlední řádě také plátno canvas sloužící pro práci s grafikou.



\subsection{Video}
Dříve nebylo možné do webových stránek vložit video tak jako dnes. K tomu účelu bylo využíváno různých zásuvných modulů třetích stran a do webových stránek se video vkládalo jak objekt. Nejvíce rozšířeným přehráváčem videí a tedy jakási náhrada za podporu videí, kterou tehdy HTML nemělo se v širším spektru stal adobe flash, který funkcí přehrávače plní v menší míře až doposud, avšak je již zastíněn efektivnějším řešením, a to právě HTML5.

Nový prvek v HTML5 vytvořený k tomuto účelu je <video>. Byl navržen tak, aby mohl být použit bez detekčních skriptů na stránce. V elementu videa je možné nastavit více souborů s videem a dle podpory si daný prohlížeč vybere podporované video. V případě, že by prohlížeč prvek videa nepodporoval, bude jej ignorovat.
\subsection{Audio}

\subsection{Plátno}
Jak již bylo avizováno výše, plátno, neboli element <canvas> slouží v HTML5 pro vykreslení grafů. herní grafiky, obrázů bitmap apod. Díky elementu <canvas> lze vykreslovat i náročnější grafické objekty zapomocí WebGl.

Plátno je bezpochyby nejdůležítější částí pro realizaci celé práce. Bude pro vyobrazení používat především <video>, zkterého bude číst data a využíje tedy elemtn video jak zdroj. Díky DOM pak dokážeme s objektem snadno manipulovat. Díky elementu <canvas> jsme tedy schopni získat kontext Javascriptového API - WebGl a díky němu schopni pracovat na úorovni grafické karty.


\section{WebGL}
Sem webgl

\section{GLSL}
GLSL


\chapter{Nikdy to nebude naprosto dokonalé}
Když jsme už napsali vše, o~čem jsme přemýšleli, uděláme si den nebo dva dny volna a~pak si přečteme sami rukopis znovu. Uděláme ještě poslední úpravy a~skončíme. Jsme si vědomi toho, že vždy zůstane něco nedokončeno, vždy existuje lepší způsob, jak něco vysvětlit, ale každá etapa úprav musí být konečná.


\chapter{Typografické a~jazykové zásady}
Při tisku odborného textu typu {\it technická zpráva} (anglicky {\it technical report}), ke kterému patří například i~text kvalifikačních prací, se často volí formát A4 a~často se tiskne pouze po jedné straně papíru. V~takovém případě volte levý okraj všech stránek o~něco větší než pravý -- v~tomto místě budou papíry svázány a~technologie vazby si tento požadavek vynucuje. Při vazbě s~pevným hřbetem by se levý okraj měl dělat o~něco širší pro tlusté svazky, protože se stránky budou hůře rozevírat a~levý okraj se tak bude oku méně odhalovat.

Horní a~spodní okraj volte stejně veliký, případně potištěnou část posuňte mírně nahoru (horní okraj menší než dolní). Počítejte s~tím, že při vazbě budou okraje mírně oříznuty.

Pro sazbu na stránku formátu A4 je vhodné používat pro základní text písmo stupně (velikosti) 11 bodů. Volte šířku sazby 15 až 16 centimetrů a~výšku 22 až 23 centimetrů (včetně případných hlaviček a~patiček). Proklad mezi řádky se volí 120 procent stupně použitého základního písma, což je optimální hodnota pro rychlost čtení souvislého textu. V~případě použití systému LaTeX ponecháme implicitní nastavení. Při psaní kvalifikační práce se řiďte příslušnými závaznými požadavky.

Stupeň písma u~nadpisů různé úrovně volíme podle standardních typografických pravidel. 
Pro všechny uvedené druhy nadpisů se obvykle používá polotučné nebo tučné písmo (jednotně buď všude polotučné nebo všude tučné). Proklad se volí tak, aby se následující text běžných odstavců sázel pokud možno na {\it pevný rejstřík}, to znamená jakoby na linky s~předem definovanou a~pevnou roztečí.

Uspořádání jednotlivých částí textu musí být přehledné a~logické. Je třeba odlišit názvy kapitol a~podkapitol -- píšeme je malými písmeny kromě velkých začátečních písmen. U~jednotlivých odstavců textu odsazujeme první řádek odstavce asi o~jeden až dva čtverčíky (vždy o~stejnou, předem zvolenou hodnotu), tedy přibližně o~dvě šířky velkého písmene M základního textu. Poslední řádek předchozího odstavce a~první řádek následujícího odstavce se v~takovém případě neoddělují svislou mezerou. Proklad mezi těmito řádky je stejný jako proklad mezi řádky uvnitř odstavce.

Při vkládání obrázků volte jejich rozměry tak, aby nepřesáhly oblast, do které se tiskne text (tj. okraje textu ze všech stran). Pro velké obrázky vyčleňte samostatnou stránku. Obrázky nebo tabulky o~rozměrech větších než A4 umístěte do písemné zprávy formou skládanky všité do přílohy nebo vložené do záložek na zadní desce.

Obrázky i~tabulky musí být pořadově očíslovány. Číslování se volí buď průběžné v~rámci celého textu, nebo -- což bývá praktičtější -- průběžné v~rámci kapitoly. V~druhém případě se číslo tabulky nebo obrázku skládá z~čísla kapitoly a~čísla obrázku/tabulky v~rámci kapitoly -- čísla jsou oddělena tečkou. Čísla podkapitol nemají na číslování obrázků a~tabulek žádný vliv.

Tabulky a~obrázky používají své vlastní, nezávislé číselné řady. Z toho vyplývá, že v~odkazech uvnitř textu musíme kromě čísla udat i~informaci o~tom, zda se jedná o~obrázek či tabulku (například ``... {\it viz tabulka 2.7} ...''). Dodržování této zásady je ostatně velmi přirozené.

Pro odkazy na stránky, na čísla kapitol a~podkapitol, na čísla obrázků a~tabulek a~v~dalších podobných příkladech využíváme speciálních prostředků DTP programu, které zajistí vygenerování správného čísla i~v~případě, že se text posune díky změnám samotného textu nebo díky úpravě parametrů sazby. Příkladem takového prostředku v~systému LaTeX je odkaz na číslo odpovídající umístění značky v~textu, například návěští ($\backslash${\tt ref\{navesti\}} -- podle umístění návěští se bude jednat o~číslo kapitoly, podkapitoly, obrázku, tabulky nebo podobného číslovaného prvku), na stránku, která obsahuje danou značku ($\backslash${\tt pageref\{navesti\}}), nebo na literární odkaz ($\backslash${\tt cite\{identifikator\}}).

Rovnice, na které se budeme v~textu odvolávat, opatříme pořadovými čísly při pravém okraji příslušného řádku. Tato pořadová čísla se píší v~kulatých závorkách. Číslování rovnic může být průběžné v~textu nebo v~jednotlivých kapitolách.

Jste-li na pochybách při sazbě matematického textu, snažte se dodržet způsob sazby definovaný systémem LaTeX. Obsahuje-li vaše práce velké množství matematických formulí, doporučujeme dát přednost použití systému LaTeX.

Mezeru neděláme tam, kde se spojují číslice s~písmeny v~jedno slovo nebo v~jeden znak -- například {\it 25krát}.

Členicí (interpunkční) znaménka tečka, čárka, středník, dvojtečka, otazník a~vykřičník, jakož i~uzavírací závorky a~uvozovky se přimykají k~předcházejícímu slovu bez mezery. Mezera se dělá až za nimi. To se ovšem netýká desetinné čárky (nebo desetinné tečky). Otevírací závorka a~přední uvozovky se přimykají k~následujícímu slovu a~mezera se vynechává před nimi -- (takto) a~``takto''.

Pro spojovací a~rozdělovací čárku a~pomlčku nepoužíváme stejný znak. Pro pomlčku je vyhrazen jiný znak (delší). V~systému TeX (LaTeX) se spojovací čárka zapisuje jako jeden znak ``pomlčka'' (například ``Brno-město''), pro sázení textu ve smyslu intervalu nebo dvojic, soupeřů a~podobně se ve zdrojovém textu používá dvojice znaků ``pomlčka'' (například ``zápas Sparta -- Slavie''; ``cena 23--25 korun''), pro výrazné oddělení části věty, pro výrazné oddělení vložené věty, pro vyjádření nevyslovené myšlenky a~v~dalších situacích (viz Pravidla českého pravopisu) se používá nejdelší typ pomlčky, která se ve zdrojovém textu zapisuje jako trojice znaků ``pomlčka'' (například ``Další pojem --- jakkoliv se může zdát nevýznamný --- bude neformálně definován v~následujícím odstavci.''). Při sazbě matematického mínus se při sazbě používá rovněž odlišný znak. V~systému TeX je ve zdrojovém textu zapsán jako normální mínus (tj. znak ``pomlčka''). Sazba v~matematickém prostředí, kdy se vzoreček uzavírá mezi dolary, zajistí vygenerování správného výstupu.

Lomítko se píše bez mezer. Například školní rok 2008/2009.

Pravidla pro psaní zkratek jsou uvedena v~Pravidlech českého pravopisu \cite{Pravidla}. I~z~jiných důvodů je vhodné, abyste tuto knihu měli po ruce. 


\section{Co to je normovaná stránka?}
Pojem {\it normovaná stránka} se vztahuje k~posuzování objemu práce, nikoliv k~počtu vytištěných listů. Z historického hlediska jde o~počet stránek rukopisu, který se psal psacím strojem na speciální předtištěné formuláře při dodržení průměrné délky řádku 60 znaků a~při 30 řádcích na stránku rukopisu. Vzhledem k~zápisu korekturních značek se používalo řádkování 2 (ob jeden řádek). Tyto údaje (počet znaků na řádek, počet řádků a~proklad mezi nimi) se nijak nevztahují ke konečnému vytištěnému výsledku. Používají se pouze pro posouzení rozsahu. Jednou normovanou stránkou se tedy rozumí 60*30 = 1800 znaků. Obrázky zařazené do textu se započítávají do rozsahu písemné práce odhadem jako množství textu, které by ve výsledném dokumentu potisklo stejně velkou plochu.

Orientační rozsah práce v~normostranách lze v~programu Microsoft Word zjistit pomocí funkce {\it Počet slov} v~menu {\it Nástroje}, když hodnotu {\it Znaky (včetně mezer)} vydělíte konstantou 1800. Do rozsahu práce se započítává pouze text uvedený v~jádru práce. Části jako abstrakt, klíčová slova, prohlášení, obsah, literatura nebo přílohy se do rozsahu práce nepočítají. Je proto nutné nejdříve označit jádro práce a~teprve pak si nechat spočítat počet znaků. Přibližný rozsah obrázků odhadnete ručně. Podobně lze postupovat i~při použití OpenOffice. Při použití systému LaTeX pro sazbu je situace trochu složitější. Pro hrubý odhad počtu normostran lze využít součet velikostí zdrojových souborů práce podělený konstantou cca 2000 (normálně bychom dělili konstantou 1800, jenže ve zdrojových souborech jsou i~vyznačovací příkazy, které se do rozsahu nepočítají). Pro přesnější odhad lze pak vyextrahovat holý text z~PDF (např. metodou cut-and-paste nebo {\it Save as Text...}) a~jeho velikost podělit konstantou 1800. 


\chapter{Závěr}
Závěrečná kapitola obsahuje zhodnocení dosažených výsledků se zvlášť vyznačeným vlastním přínosem studenta. Povinně se zde objeví i zhodnocení z pohledu dalšího vývoje projektu, student uvede náměty vycházející ze zkušeností s řešeným projektem a uvede rovněž návaznosti na právě dokončené projekty.

%=========================================================================
