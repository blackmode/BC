% Tento soubor nahraďte vlastním souborem s přílohami (nadpisy níže jsou pouze pro příklad)
% This file should be replaced with your file with an appendices (headings below are examples only)

% Umístění obsahu paměťového média do příloh je vhodné konzultovat s vedoucím
% Placing of table of contents of the memory media here should be consulted with a supervisor
%\chapter{Obsah přiloženého paměťového média}

%\chapter{Manuál}

%\chapter{Konfigurační soubor} % Configuration file

%\chapter{RelaxNG Schéma konfiguračního souboru} % Scheme of RelaxNG configuration file

%\chapter{Plakát} % poster


\section*{Vytvoření a běh programu}
Po vytvoření kontextu \texttt{webgl} je již možné využívat funkce pro práci s WebGL. Funkce by se daly rozdělit na několik částí, nejprve ty, které pracují s shadery, dále ty které vytváří program, v neposlední řadě funkce pro prací s buffery a vykreslením dat. Všechny funkce WebGL pracují nad vytvořeným kontextem plátna\footnote{Vytvořený WebGL kontext budeme označovat jako \texttt{gl}}.

\begin{description}
	\item[\texttt{gl.createShader(type)}] - vytvoří objekt shaderu, vstupním argumenty \texttt{type} mohou být \\ \texttt{gl.VERTEX\_SHADER} nebo \texttt{gl.FRAGMENT\_SHADER}
	
	\item[\texttt{gl.shaderSource(shader, source)}] - asociuje objekt shaderu \texttt{shader} se zdrojovým kódem samotného shaderu \texttt{source} v textové formě
	
	\item[\texttt{gl.compileShader(shader)}] - kompiluje GLSL kód na binární data, vstupním parametrem je objekt shaderu \texttt{shader}
	
\end{description}

Následuje práce s programem. Funkce \texttt{gl.createProgram()} vytvoří objekt programu. Poté \texttt{gl.attachShader(program, shader)} - přiloží daný shader k programu.
Funkce \\ \texttt{gl.linkProgram(program)}, která slinkuje vše dohromady, \texttt{gl.useProgram(program)} - nastaví daný webgl program jako část současného vykreslovacího stavu.

\begin{description}
	
	\item[\texttt{gl.getAttribLocation(program, attribute)}] - vrací referenci na atribut v ve vertex shaderu
	
	\item[\texttt{gl.enableAttribute(attribute)}] - aktivuje atribut 
	
	\item[\texttt{gl.createBuffer()}] - vytvoří nový buffer objekt pro data
	
	\item[\texttt{gl.bindBuffer(target, buffer)}] - nastaví vytvořený buffer \texttt{buffer} jako aktivní, parametr \texttt{target} specifikuje data s jakými se bude pracovat
	
	\item[\texttt{gl.bufferData(target, srcData, usage)}] - parametr \texttt{target} má stejný význam jako u funkce \texttt{bindBuffer}, \texttt{srcData} určuje data, která se nahrají do bufferu, poslední parametr \texttt{usage} specifikuje způsob využiti uložených dat
	
	\item[\texttt{gl.vertexAttribPointer(attrib, size, type, normalized, stride, offset)}] -  \texttt{attrib} prováže atribut programu s aktivním bufferem, \texttt{size} určuje, z kolika prvků se skládají data v bufferu, \texttt{type} spcecifikuje datový typ dat v bufferu, \texttt{normalized} určuje zdali budou hodnoty normalizovány, parametr je typu \texttt{bool}, dále \texttt{stride} udávající zdali budeme nějaké hodnoty přeskakovat, \texttt{offset} odkud budeme data číst.
	
	
\end{description}

Samotné vykreslení uložených data v bufferech se realizuje funkcemi:


\begin{description}
	
	\item[\texttt{gl.drawArrays(mode, first, count)}] - vykresluje vstupní data. Parametr \texttt{mode} určuje, jaké grafické primitivum budem použito pro vykreslení, \texttt{first} určuje odkud se data zažnou upracovávat, \texttt{count} specifikuje počet vykreslovaných hodnot
	
	\item[\texttt{gl.drawElements(mode, count, type, offset)}] - stejně jako u \texttt{drawArrays()} určuje vykreslované grafické primitivum, to stejné se týká i parametru \texttt{count}, \texttt{type} definuje typ dat indexů a \texttt{offset} odkud začínáme zpracovávat indexy. HLavním rozdílem vůči funkci \texttt{drawArrays()} je ten, že \texttt{gl.drawElements(...)} nevykresluje body přímo, ale dostává jako vstupní data indexy na jednotlivé body, které vykresluje funkce \texttt{drawArrays()}
	
	\end{description}

\newpage

\chapter{Jak pracovat s touto šablonou}
\label{jak}

V této kapitole je uveden popis jednotlivých částí šablony, po kterém následuje stručný návod, jak s touto šablonou pracovat. 

Jedná se o přechodnou verzi šablony. Nová verze bude zveřejněna do konce roku 2016 a bude navíc obsahovat nové pokyny ke správnému využití šablony, závazné pokyny k~vypracování bakalářských a diplomových prací (rekapitulace pokynů, které jsou dostupné na~webu) a nezávazná doporučení od vybraných vedoucích. Jediné soubory, které se v nové verzi změní, budou projekt-01-kapitoly-chapters.tex a projekt-30-prilohy-appendices.tex, jejichž obsah každý student vymaže a nahradí vlastním. Šablonu lze tedy bez problémů využít i~v~současné verzi.

\section*{Popis částí šablony}

Po rozbalení šablony naleznete následující soubory a adresáře:
\begin{DESCRIPTION}
  \item [bib-styles] Styly literatury (viz níže). 
  \item [obrazky-figures] Adresář pro Vaše obrázky. Nyní obsahuje placeholder.pdf (tzv. TODO obrázek, který lze použít jako pomůcku při tvorbě technické zprávy), který se s prací neodevzdává. Název adresáře je vhodné zkrátit, aby byl jen ve zvoleném jazyce.
  \item [template-fig] Obrázky šablony (znak VUT).
  \item [fitthesis.cls] Šablona (definice vzhledu).
  \item [Makefile] Makefile pro překlad, počítání normostran, sbalení apod. (viz níže).
  \item [projekt-01-kapitoly-chapters.tex] Soubor pro Váš text (obsah nahraďte).
  \item [projekt-20-literatura-bibliography.bib] Seznam literatury (viz níže).
  \item [projekt-30-prilohy-appendices.tex] Soubor pro přílohy (obsah nahraďte).
  \item [projekt.tex] Hlavní soubor práce -- definice formálních částí.
\end{DESCRIPTION}

Výchozí styl literatury (czechiso) je od Ing. Martínka, přičemž anglická verze (englishiso) je jeho překladem s drobnými modifikacemi. Oproti normě jsou v něm určité odlišnosti, ale na FIT je dlouhodobě akceptován. Alternativně můžete využít styl od Ing. Radima Loskota nebo od Ing. Radka Pyšného\footnote{BP Ing. Radka Pyšného \url{http://www.fit.vutbr.cz/study/DP/BP.php?id=7848}}. Alternativní styly obsahují určitá vylepšení, ale zatím nebyly řádně otestovány větším množstvím uživatelů. Lze je považovat za beta verze pro zájemce, kteří svoji práci chtějí mít dokonalou do detailů a neváhají si nastudovat detaily správného formátování citací, aby si mohli ověřit, že je vysázený výsledek v pořádku.

Makefile kromě překladu do PDF nabízí i další funkce:
\begin{itemize}
  \item přejmenování souborů (viz níže),
  \item počítání normostran,
  \item spuštění vlny pro doplnění nezlomitelných mezer,
  \item sbalení výsledku pro odeslání vedoucímu ke kontrole (zkontrolujte, zda sbalí všechny Vámi přidané soubory, a případně doplňte).
\end{itemize}

Nezapomeňte, že vlna neřeší všechny nezlomitelné mezery. Vždy je třeba manuální kontrola, zda na konci řádku nezůstalo něco nevhodného -- viz Internetová jazyková příručka\footnote{Internetová jazyková příručka \url{http://prirucka.ujc.cas.cz/?id=880}}.

\paragraph {Pozor na číslování stránek!} Pokud má obsah 2 strany a na 2. jsou jen \uv{Přílohy} a~\uv{Seznam příloh} (ale žádná příloha tam není), z nějakého důvodu se posune číslování stránek o 1 (obsah \uv{nesedí}). Stejný efekt má, když je na 2. či 3. stránce obsahu jen \uv{Literatura} a~je možné, že tohoto problému lze dosáhnout i jinak. Řešení je několik (od~úpravy obsahu, přes nastavení počítadla až po sofistikovanější metody). \textbf{Před odevzdáním proto vždy překontrolujte číslování stran!}


\section*{Doporučený postup práce se šablonou}

\begin{enumerate}
  \item \textbf{Zkontrolujte, zda máte aktuální verzi šablony.} Máte-li šablonu z předchozího roku, na stránkách fakulty již může být novější verze šablony s~aktualizovanými informacemi, opravenými chybami apod.
  \item \textbf{Zvolte si jazyk}, ve kterém budete psát svoji technickou zprávu (česky, slovensky nebo anglicky) a svoji volbu konzultujte s vedoucím práce (nebyla-li dohodnuta předem). Pokud Vámi zvoleným jazykem technické zprávy není čeština, nastavte příslušný parametr šablony v souboru projekt.tex (např.: \verb|documentclass[english]{fitthesis}| a přeložte prohlášení a poděkování do~angličtiny či slovenštiny.
  \item \textbf{Přejmenujte soubory.} Po rozbalení je v šabloně soubor projekt.tex. Pokud jej přeložíte, vznikne PDF s technickou zprávou pojmenované projekt.pdf. Když vedoucímu více studentů pošle projekt.pdf ke kontrole, musí je pracně přejmenovávat. Proto je vždy vhodné tento soubor přejmenovat tak, aby obsahoval Váš login a (případně zkrácené) téma práce. Vyhněte se však použití mezer, diakritiky a speciálních znaků. Vhodný název tedy může být např.: \uv{xlogin00-Cisteni-a-extrakce-textu.tex}. K přejmenování můžete využít i přiložený Makefile:
\begin{verbatim}
make rename NAME=xlogin00-Cisteni-a-extrakce-textu
\end{verbatim}
  \item Vyplňte požadované položky v souboru, který byl původně pojmenován projekt.tex, tedy typ, rok (odevzdání), název práce, svoje jméno, ústav (dle zadání), tituly a~jméno vedoucího, abstrakt, klíčová slova a další formální náležitosti.
  \item Nahraďte obsah souborů s kapitolami práce, literaturou a přílohami obsahem svojí technické zprávy. Jednotlivé přílohy či kapitoly práce může být výhodné uložit do~samostatných souborů -- rozhodnete-li se pro toto řešení, je doporučeno zachovat konvenci pro názvy souborů, přičemž za číslem bude následovat název kapitoly. 
  \item Nepotřebujete-li přílohy, zakomentujte příslušnou část v projekt.tex a příslušný soubor vyprázdněte či smažte. Nesnažte se prosím vymyslet nějakou neúčelnou přílohu jen proto, aby daný soubor bylo čím naplnit. Vhodnou přílohou může být obsah přiloženého paměťového média.
  \item Nascanované zadání uložte do souboru zadani.pdf a povolte jeho vložení do práce parametrem šablony v projekt.tex (\verb|documentclass[zadani]{fitthesis}|).
  \item Nechcete-li odkazy tisknout barevně (tedy červený obsah -- bez konzultace s vedoucím nedoporučuji), budete pro tisk vytvářet druhé PDF s tím, že nastavíte parametr šablony pro tisk: (\verb|documentclass[zadani,print]{fitthesis}|).  Barevné logo se nesmí tisknout černobíle!
  \item Vzor desek, do kterých bude práce vyvázána, si vygenerujte v informačním systému fakulty u zadání. Pro disertační práci lze zapnout parametrem v šabloně (více naleznete v souboru fitthesis.cls).
  \item Nezapomeňte, že zdrojové soubory i (obě verze) PDF musíte odevzdat na CD či jiném médiu přiloženém k technické zprávě.
\end{enumerate}

\subsection*{Pokyny pro oboustranný tisk}
\begin{itemize}
\item Zapíná se parametrem šablony: \verb|\documentclass[twoside]{fitthesis}|
\item Po vytištění oboustranného listu zkontrolujte, zda je při prosvícení sazební obrazec na obou stranách na stejné pozici. Méně kvalitní tiskárny s duplexní jednotkou mají často posun o 1--3 mm. Toto může být u některých tiskáren řešitelné tak, že vytisknete nejprve liché stránky, pak je dáte do stejného zásobníku a vytisknete sudé.
\item Za titulním listem, obsahem, literaturou, úvodním listem příloh, seznamem příloh a případnými dalšími seznamy je třeba nechat volnou stránku, aby následující část začínala na liché stránce (\textbackslash cleardoublepage).
\item  Konečný výsledek je nutné pečlivě překontrolovat.
\end{itemize}


\subsection*{Užitečné nástroje}
\label{nastroje}

Následující seznam není výčtem všech využitelných nástrojů. Máte-li vyzkoušený osvědčený nástroj, neváhejte jej využít. Pokud však nevíte, který nástroj si zvolit, můžete zvážit některý z následujících:

\begin{description}
	\item[\href{http://miktex.org/download}{MikTeX}] \LaTeX{} pro Windows -- distribuce s jednoduchou instalací a vynikající automatizací stahování balíčků.
	\item[\href{http://texstudio.sourceforge.net/}{TeXstudio}] Přenositelné opensource GUI pro \LaTeX{}.  Ctrl+klik umožňuje přepínat mezi zdrojovým textem a PDF. Má integrovanou kontrolu pravopisu, zvýraznění syntaxe apod. Pro jeho využití je nejprve potřeba nainstalovat MikTeX.
	\item[\href{http://jabref.sourceforge.net/download.php}{JabRef}] Pěkný a jednoduchý program v Javě pro správu souborů s bibliografií (literaturou). Není potřeba se nic učit -- poskytuje jednoduché okno a formulář pro editaci položek.
	\item[\href{https://inkscape.org/en/download/}{InkScape}] Přenositelný opensource editor vektorové grafiky (SVG i PDF). Vynikající nástroj pro tvorbu obrázků do odborného textu. Jeho ovládnutí je obtížnější, ale výsledky stojí za to.
	\item[\href{https://git-scm.com/}{GIT}] Vynikající pro týmovou spolupráci na projektech, ale může výrazně pomoci i jednomu autorovi. Umožňuje jednoduché verzování, zálohování a přenášení mezi více počítači.
	\item[\href{http://www.overleaf.com/}{Overleaf}] Online nástroj pro \LaTeX{}. Přímo zobrazuje náhled a umožňuje jednoduchou spolupráci (vedoucí může průběžně sledovat psaní práce), vyhledávání ve zdrojovém textu kliknutím do PDF, kontrolu pravopisu apod. Zdarma jej však lze využít pouze s určitými omezeními (někomu stačí na disertaci, jiný na ně může narazit i při psaní bakalářské práce) a pro dlouhé texty je pomalejší.
\end{description}

\subsection*{Užitečné balíčky pro \LaTeX}

Studenti při sazbě textu často řeší stejné problémy. Některé z nich lze vyřešit následujícími balíčky pro \LaTeX:

\begin{itemize}
  \item \verb|amsmath| -- rozšířené možnosti sazby rovnic,
  \item \verb|float, afterpage, placeins| -- úprava umístění obrázků,
  \item \verb|fancyvrb, alltt| -- úpravy vlastností prostředí Verbatim, 
  \item \verb|makecell| -- rozšíření možností tabulek,
  \item \verb|pdflscape, rotating| -- natočení stránky o 90 stupňů (pro obrázek či tabulku),
  \item \verb|hyphenat| -- úpravy dělení slov,
  \item \verb|picture, epic, eepic| -- přímé kreslení obrázků.
\end{itemize}

Některé balíčky jsou využity přímo v šabloně (v dolní části souboru fitthesis.cls). Nahlédnutí do jejich dokumentace může být rovněž užitečné.

Sloupec tabulky zarovnaný vlevo s pevnou šířkou je v šabloně definovaný \uv{L} (používá se jako \uv{p}).

